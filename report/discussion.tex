\section{Discussion}
\subsection{Analog}
\subsubsection{Process variation}

The values for $R_{DS}$ for a switch transistor turned off is $5 \mathrm{G\Omega}$ in FF, $500 \mathrm{G\Omega}$ in TT and $50 \mathrm{T\Omega}$. For a voltage of $1\mathrm{V}$ this means a leakage of $200\mathrm{pA}$ in FF, and that is actually quite dramatic in our circuit. In SS however, a voltage of $1\mathrm{V}$ means a leakage current of $20\mathrm{fA}$, and that is actually better for our circuit than the TT case.

The $R_{DS}$ values when the transistors are switched on, however are not varying that much. The $R_{DS}$ values for FF, TT and SS respectively are $500 \mathrm{k\Omega}$, $750 \mathrm{k\Omega}$ and $1 \mathrm{M\Omega}$. To conduct a current of $1\mathrm{nA}$, the required voltages are respectively $500 \mathrm{\mu V}$, $750 \mathrm{\mu V}$ and $1 \mathrm{mV}$. All these values are very small compared to $V_{DD}$, so when switched on, our transistors can be considered ideal conductors for practical purposes.

For the full analog circuit, we can see as expected that the way in which the FF corner increases leakage current has a dramatic impact on the system as a whole, while the way SS increases the on resistance does not make a big change. And the cool thing is that SS also reduces leakage current compared to TT, so that SS in total improves behaviour of our design.
