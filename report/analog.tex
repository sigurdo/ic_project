\section{Analog pixel circuit design.}

In this circuit we will design the analog part of the camera, also called the pixel circuit.

\subsection{Choosing dimensions for switch transistors}
The transistors M1, M2 and M4 all function as switches, where a digital gate input decides whether the transistor should act as a short-circuit between drain and source, or as an open circuit. Since these transistors simply should have two possible states, it is key that the leakage current is minimized. This is particularly important for M1 and M2 to ensure that the voltage over $C_s$ is as constant as possible during readout. In \emph{Analog Circuit Design} by Tony Chan Carusone, one can read in section 1.4.1 that the subthreshold leakage current is given by

\begin{equation}
    \label{eq:leakage}
    I_{off} = (n-1) \mu_n C_{ox} \left( \frac{W}{L} \right) \left( \frac{kT}{q} \right)^2 \exp{(-qV_t / nkT)}.
\end{equation}

In order to minimize this leakage current, $\frac{W}{L}$ need to be as small as possible, meaning the smallest possible width $W$ and the largest possible length $L$ is desired. The transistor technology used limits the width to 

\begin{equation}
    \label{limitsW}
    1.08 \mu\textrm{m} \leq W \leq 5.04 \mu\textrm{m}
\end{equation}

and limits the length Tony

\begin{equation}
    \label{limitsL}
    0.36 \mu\textrm{m} \leq L \leq 1.08 \mu\textrm{m}.
\end{equation}

\subsection{Choosing value for CS}

To choose a suitable value for CS spice simulations can be used. To know what values are the best we can for example look at the four corner cases for exposure time and light conditions. The corner cases will be denoted exposure-light, so for example max-min is maximum exposure time and minimum light. Each corner simulation in spice produces a graph that shows the voltage over CS as a function of exposure time, and the value we are interested in is the voltage over CS at the end of the expose time. In the graph that means 3ms or 31ms since each simulation start with 1ms of erasing. There are many possible approaches to how these corners should be tuned, and the one that we have chosen here is the following:

\begin{itemize}
    \item Max-max corner should make CS fully charged.
    \item Min-min corner should leave CS uncharged.
    \item Min-max and max-min corners should make CS half full charged.
\end{itemize}

The reason for why we want the corners to be like that is beyond the scope of this report.

\subsection{Choosing values for M3 and active load}
